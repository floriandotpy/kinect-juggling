\documentclass[dvipsnames]{beamer}
\usepackage[utf8]{inputenc}
\usepackage[T1]{fontenc}
\usepackage{lmodern}
\usepackage[ngerman]{babel}
\usepackage{graphicx}
\usepackage{listings}
\lstset{
    %backgroundcolor=\color[rgb]{0.95,0.95,0.95},
    columns=flexible, 
    numbers=left, 
    numbersep=8pt, 
    xleftmargin=5pt,
    numberstyle=\scriptsize\color[rgb]{0.700,0.700,0.700}, 
    basicstyle=\ttfamily\scriptsize,
    captionpos=b,
    %keywordstyle=\color{Mahogany},
    commentstyle=\color[rgb]{0.133,0.545,0.133},
    stringstyle=\color{MidnightBlue},
    breaklines=true,
    frame=shadowbox,
    rulesepcolor=\color[rgb]{0.7,0.7,0.7} % color of box-shadow
}
\usepackage{verbatim} % mainly for multi-line comments

\resetcounteronoverlays{lstlisting}
\usepackage{tikz}
\setbeamercovered{highly dynamic}

\title{Speeding up Python using Cython}
\author[Rolf $\cdot$ Thiemo $\cdot$ Flo]{Rolf Boomgaarden $\cdot$ Thiemo Gries $\cdot$ Florian Letsch}
\institute{Universität Hamburg}
\date{November 28th, 2013}

\begin{document}

\frame
{
\titlepage
}

%\frame{\tableofcontents}

\section{Introducing Cython}
\begin{frame}{What is Cython?}
\begin{itemize}
\item Compiler, compiles Python-like code to C-code
\item Code is still executed in the Python runtime environment
\item But is compiled to native machine code instead of Python bytecode
\item Can result in more speed and easy wrapping of C libraries
\end{itemize}
%FIXME bullet points
\end{frame}

\begin{comment}
\begin{frame}[fragile]{Hello World}
\begin{lstlisting}[language=Python]
print "Hello World"
\end{lstlisting}

\begin{lstlisting}[language=Python]
from distutils.core import setup
from Cython.Build import cythonize

setup(
    ext_modules = cythonize("helloworld.pyx")
)
\end{lstlisting}

\begin{lstlisting}[language=Python]
$ python setup.py build_ext --inplace
\end{lstlisting}

\begin{lstlisting}[language=Python]
>>> import helloworld
Hello World
\end{lstlisting}

\end{frame}
\end{comment}

\begin{frame}[fragile]{Cython workflow}{first approach}
\setbeamercovered{transparent}
\begin{itemize}
\item<1-> Write a helloworld.pyx source file
\begin{lstlisting}[language=Python]
print "Hello World"
\end{lstlisting}
\item<1-> Run the Cython compiler to generate a C file\\
\begin{lstlisting}[language=Python]
$ cython helloworld.pyx
\end{lstlisting}
%{\tt \$ cython helloworld.pyx}
\item<2-> Run a C compiler to generate a compiled library\\
\begin{lstlisting}[language=Python]
$ gcc [...] -o helloworld.so helloworld.c
\end{lstlisting}
%{\tt \$ gcc [...] -o helloworld.so helloworld.c}
\item<2-> Run the Python interpreter and import the module\\
\begin{lstlisting}[language=Python]
>>> import helloworld
Hello World
\end{lstlisting}
\end{itemize}
%FIXME
\end{frame}

\begin{frame}[fragile]{Cython workflow}{second approach}
\setbeamercovered{transparent}
\begin{itemize}
\item<1-> Write a helloworld.pyx source file
\item<1-> Write a setup.py file with compile information
\begin{lstlisting}[language=Python]
from distutils.core import setup
from Cython.Build import cythonize

setup(
    ext_modules = cythonize("helloworld.pyx")
)
\end{lstlisting}
\item<2-> Let Python compile the file
\begin{lstlisting}[language=Python]
$ python setup.py build_ext --inplace
\end{lstlisting}
\item<2-> Run the Python interpreter and import the module\\
\end{itemize}
\end{frame}

\begin{frame}[fragile]{Cython workflow}{third approach: pyximport}
\begin{itemize}
\item Write a helloworld.pyx source file
\item Use Pyximport
\begin{lstlisting}[language=Python]
>>> import pyximport; pyximport.install()
>>> import helloworld
Hello World
\end{lstlisting}
\end{itemize}
\end{frame}

\section{Image processing}

\begin{frame}{}
%\color{red}FIXME cool image which represents something nice and funny
\begin{center}\includegraphics[scale=0.42]{challenge.jpg}\end{center}
\end{frame}

\begin{frame}[fragile]{Pure Python}
\begin{lstlisting}[language=Python,caption={add1.py}]
import numpy as np

def my_add(a, b):
    (...) # validate parameter
	
    dtype = a.dtype
    height = a.shape[0]
    width = a.shape[1]

    result = np.zeros((height, width), dtype=dtype)

    for y in range(height):
        for x in range(width):
            result[y,x] = a[y,x] + b[y,x]

    return result
\end{lstlisting}
\pause
{\tt Time: $\sim$19 minutes (2048x2048, 100x)}
\end{frame}

\begin{frame}[fragile]{Python run through Cython}
\begin{lstlisting}[language=Python,caption={add2.pyx}]
import numpy as np

def my_add(a, b):
    (...) # validate parameter
	
    dtype = a.dtype
    height = a.shape[0]
    width = a.shape[1]

    result = np.zeros((height, width), dtype=dtype)

    for y in range(height):
        for x in range(width):
            result[y,x] = a[y,x] + b[y,x]

    return result
\end{lstlisting}
\pause
{\tt Time: $\sim$16 minutes (2048x2048, 100x)}
\end{frame}

\begin{frame}[fragile]{Cython: Adding types}

\begin{lstlisting}[language=Python,caption={add3.pyx}]
import numpy as np; cimport numpy as np

DTYPE = np.uint8
ctypedef np.uint8_t DTYPE_t

def my_add(np.ndarray a, np.ndarray b):
    (...) # validate parameter

    cdef int height = a.shape[0]
    cdef int width = a.shape[1]
    cdef np.ndarray result = np.zeros((height, width), dtype=DTYPE)

    cdef int x, y
    for y in range(height):
        for x in range(width):
            result[y,x] = a[y,x] + b[y,x]

    return result
\end{lstlisting}
\pause
\vspace{-0.2cm}
{\tt Time: $\sim$16 minutes (2048x2048, 100x)}
\end{frame}

\begin{frame}[fragile]{Cython: Efficient indexing}
\begin{lstlisting}[language=Python,caption={add4.pyx}]
import numpy as np; cimport numpy as np

DTYPE = np.uint8
ctypedef np.uint8_t DTYPE_t

def my_add(np.ndarray[DTYPE_t,ndim=2] a, np.ndarray[DTYPE_t,ndim=2] b):
    (...) # validate parameter

    cdef int height = a.shape[0]
    cdef int width = a.shape[1]
    cdef np.ndarray[DTYPE_t, ndim=2] result = np.zeros((height, width), dtype=DTYPE)

    cdef int x, y
    for y in range(height):
        for x in range(width):
            result[y,x] = a[y,x] + b[y,x]

    return result
\end{lstlisting}
\pause
\vspace{-0.2cm}
{\tt Time: 1.249 seconds (2048x2048, 100x)}
\end{frame}

\begin{frame}[fragile]{Cython: Don't check boundaries}
\begin{lstlisting}[language=Python,caption={add5.pyx}]
import numpy as np; cimport numpy as np; cimport cython

DTYPE = np.uint8
ctypedef np.uint8_t DTYPE_t

@cython.boundscheck(False)
def my_add(np.ndarray[DTYPE_t,ndim=2] a, np.ndarray[DTYPE_t,ndim=2] b):
    (...) # validate parameter

    cdef int height = a.shape[0]
    cdef int width = a.shape[1]
    cdef np.ndarray[DTYPE_t, ndim=2] result = np.zeros((height, width), dtype=DTYPE)

    cdef int x, y
    for y in range(height):
        for x in range(width):
            result[y,x] = a[y,x] + b[y,x]

    return result
\end{lstlisting}
\pause
\vspace{-0.3cm}
{\tt Time: 1.086 seconds (2048x2048, 100x)}
\end{frame}

\section{Speedup}

\begin{comment}
\begin{frame}[fragile]{Cython vs numpy}

\Huge \begin{center}a + b\end{center}

\pause
{\tt\normalsize Time: 0.376 seconds (2048x2048, 100x)}
\end{frame}
\end{comment}

\section{More features?}

\begin{frame}{Obtaining speedup}
\begin{itemize}
\item Static typing
\item Access data buffer directly at C level
\item Cython gimmicks
\end{itemize}
\end{frame}

\begin{frame}{Conclusion}

\begin{tikzpicture}[x=1.4cm,y=0.15pt]
\centering\hspace{1cm}
% Bars
\draw[ycomb, color=blue!50, line width=1.2cm]
plot coordinates {(0,1157) (1,971) (2,996)};
% Axes
\draw (-0.5,0) -- (-0.5,1000);
\draw (-0.5,0) -- (2.5,0);
\draw (0,-50) node {\footnotesize py};
\draw (1,-50) node {\footnotesize py as cy};
\draw (2,-50) node {\footnotesize typed};
\draw[color=white] (0, 1057) node {1157};
\draw[color=white] (1, 871) node {971};
\draw[color=white] (2, 896) node {996};

% Bars
\draw[ycomb, color=blue!50, line width=1.2cm]
plot coordinates {(3.4,125) (4.4,109)};
% Axes
%\draw (2.8,0) -- (2.8,1000);
\draw (2.8,0) -- (6,0);
\draw (3.4,-50) node {\footnotesize indices};
\draw (4.4,-50) node {\footnotesize bounds};
\draw[color=white] (3.4, 80) node {1.25};
\draw[color=white] (4.4, 65) node {1.09};
\end{tikzpicture}

\end{frame}


\begin{frame}{Conclusion}

\begin{tikzpicture}[x=1.4cm,y=0.15pt]
\centering\hspace{1cm}
% Bars
\draw[ycomb, color=blue!50, line width=1.2cm]
plot coordinates {(0,1157) (1,971) (2,996)};
% Axes
\draw (-0.5,0) -- (-0.5,1000);
\draw (-0.5,0) -- (2.5,0);
\draw (0,-50) node {\footnotesize py};
\draw (1,-50) node {\footnotesize py as cy};
\draw (2,-50) node {\footnotesize typed};
\draw[color=white] (0, 1057) node {1157};
\draw[color=white] (1, 871) node {971};
\draw[color=white] (2, 896) node {996};

% Bars
\draw[ycomb, color=blue!50, line width=1.2cm]
plot coordinates {(3.4,125) (4.4,109) (5.4,38)};
% Axes
%\draw (2.8,0) -- (2.8,1000);
\draw (2.8,0) -- (6,0);
\draw (3.4,-50) node {\footnotesize indices};
\draw (4.4,-50) node {\footnotesize bounds};
\draw (5.4,-50) node {\footnotesize numpy};
\draw[color=white] (3.4, 80) node {1.25};
\draw[color=white] (4.4, 65) node {1.09};
\draw[color=blue!50] (5.4, 80) node {0.38};
\end{tikzpicture}

\end{frame}

\begin{frame}{Now what?}
\begin{enumerate}
\item http://docs.cython.org (tutorial style plus reference)
\pause\item \lstinline{with nogil:} release GIL
\item \lstinline{cython.parallel}, parallelism with OpenMP support
\item PyPy support (considered mostly usable since Cython 0.17)
\pause\item actual speedup compared to CPython; but compared to PyPy etc. ?
\item can we juggle with this? Pretty sure we can!
\end{enumerate}
\end{frame}

\begin{frame}{Now back to work}

\begin{center}\includegraphics[scale=0.7]{superman-coding.jpg}\end{center}

\end{frame}

\end{document}






